%%
% このファイルは筑波大学情報学群情報科学類の卒業研究論文のサンプルです。
% このファイルを書き換えて、このサンプルと同様の書式の論文をLaTeXを使って
% 作成できます。
%
% OSやLaTeXの設定によっては漢字コードや改行コードを変更する必要があります。
%%
\documentclass[a4paper,11pt]{jreport}

%%【PDF, PostScript, JPEG, PNG等の画像の貼り込み】
%% dvipdfmx を使う場合
\usepackage[dvipdfmx]{graphicx}
%% dvipdfmx を使ってPDFの「しおり」を付ける場合
%%\usepackage[dvipdfmx,bookmarks=true,bookmarksnumbered=true,bookmarkstype=toc]{hyperref} \usepackage{pxjahyper}
\usepackage{ulem}
\usepackage{times} % use Times font instead of default one

\setcounter{tocdepth}{3}
\setcounter{page}{-1}

\setlength{\oddsidemargin}{0.1in}
\setlength{\evensidemargin}{0.1in}
\setlength{\topmargin}{0in}
\setlength{\textwidth}{6in}
%\setlength{\textheight}{10.1in}
\setlength{\parskip}{0em}
\setlength{\topsep}{0em}

%% タイトル生成用パッケージ(重要)
\usepackage{coins}

%% タイトル
\title{TODO: タイトル}
%% 著者
\author{岡部 純弥}
%% 指導教員
\advisor{岡 瑞起, 阿部 洋丈}

%% 年度と主専攻名
\fiscalyear{2023}
\majorfield{ソフトウェアサイエンス主専攻}

\begin{document}
\maketitle
\thispagestyle{empty}
\newpage

\thispagestyle{empty}
\vspace*{20pt plus 1fil}
\parindent=1zw
\noindent
%%
%% 論文の要旨
%%
\begin{center}
{\Large \bf 要  旨}
\vspace{2cm}
\end{center}

優れた輻輳制御アルゴリズムを発見することは難しい。主な理由として、コンピュータネットワークの構造が時々刻々と変化することが挙げられる。つまり、特定のネットワーク構造化で最適なアルゴリズムを探索しても、時間経過につれ、良いパフォーマンスを発揮できなくなってしまう。

そこで、大規模言語モデルと進化アルゴリズムを用いることで任意のネットワーク環境下で最適化を行う探索手法を提案する。既存の探索手法では、探索空間の制約が厳しかったものの、大規模言語モデルを用いることで、この問題を解決できた。

TODO: 事実確認

実際にネットワークシミュレータを用いた実験を行い、ベンチマークを超える輻輳制御アルゴリズムを発見できた。

%%%%%
\par
\vspace{0pt plus 1fil}
\newpage

\pagenumbering{roman} % I, II, III, IV
\tableofcontents
\listoffigures
%\listoftables

\pagebreak \setcounter{page}{1}
\pagenumbering{arabic} % 1,2,3

\chapter{序論}

\section{研究背景}

TCP/IPネットワークでは、トランスポート層で輻輳制御が行われている。輻輳制御の難しさとして、限られた観測可能な値から、観測不可能な状態を推定しなければならないことが挙げられる。
つまり、限られた観測値から、輻輳が発生しているのかを判断したうえで、パケット送信量を制御しなければなければならない。

さらに、ネットワーク構造が変化し続けているため、支配的なアルゴリズムが存在することもなく、数年に一度はパラダイムシフトが起こっている。

現在の輻輳制御アルゴリズムの探索、発見は、ヒューリスティックに行われている側面があり、人的リソースを割き続けなければらない。つまり、ネットワーク構造の変化に柔軟に対応可能な探索手法の発見が求められている。

\newpage

\section{本論文の構成}

第二章の前半では、輻輳制御アルゴリズムの概要や、その発展について述べる。
第二章の後半では、近年の大規模言語モデルの発展や、その応用について概説する。特に、輻輳制御アルゴリズムの探索手法として、大規模言語モデルを用いることの可能性について述べる。
第三章では、これらの関連研究を踏まえたうえでの仮説、および提案手法について述べる。
第四章では、仮説を検証するために提案手法の実装、および実験の詳細について述べる。
第五章では、実験結果を示し、その結果に対する考察を行う。
第六章では、本論文のまとめと今後の課題について述べる。

\newpage


\chapter{形式}

ここでは、論文の表紙および本体の記述方法について述べる。

\section{表紙}

表紙は、以下の各項目に相当する文字列を記述した上で、\texttt{$\backslash$maketitle}により作成する。

\begin{description} \parskip=1pt
\item{題目: }
題目は\texttt{$\backslash$title}に記述する。行替えを行う場合には $\backslash \backslash$ を入力する。
ただし、題目の最後に$\backslash \backslash$ を入力するとコンパイルが通らなくなるので注意する。
なお、題目が複数行に渡るなどの理由により表紙ページがあふれた場合にはスタイルファイルを変更する必要がある。
\item{著者名: }
著者名は\texttt{$\backslash$author}に記述する。
\item{指導教員名: }
指導教員名は\texttt{$\backslash$advisor}に記述する。
2名以上の場合には複数名を記述する。
\item{主専攻名: }
主専攻名は\texttt{$\backslash$majorfield}に記述する。「○○主専攻」という形式にすること。
\item{年度: }
年度は\texttt{$\backslash$fiscalyear} に記述する。年度は提出時のものを記述すること。
\end{description}

\section{本体}

本体は1段組で記述する。

図表には番号と説明(caption)を付け、文章中で参照する。
表~\ref{table:scores}と図~\ref{figure:smile}はそれぞれ
表と図の例である。表の説明は表の上に、図の説明は図の下に書くことが多い。
図の挿入に用いる \LaTeX のパッケージについては使用環境に合わせて自由に選択してほしい。

\begin{table}[hbt]
\caption{表の例}
\label{table:scores}
\begin{center}
\begin{tabular}{|c|r|r|r|r|}
\hline
年度 & 1年次 & 2年次 & 3年次 & 4年次 \\
\hline
2016 & 85 & 92 & 86 &  88 \\
2017 & 83 & 89 & 90 & 102 \\
2018 & 88 & 87 & 91 & 112 \\
\hline
\end{tabular}
\end{center}
\end{table}
\medskip

レポートや論文の書き方、日本語の\LaTeX の使い方に関しては、Web 上の情報や
参考書など~\cite{Bibunsho,ScienceResearchWriting}を参照のこと。
また、参考文献、図、表の入れ方を含む、文章のスタイルについては、
ACM, IEEE, 情報処理学会, 電子情報通信学会などの学会が出版している
ジャーナルや国際会議の論文のスタイルを参考にするとよい。

\chapter*{謝辞}
\addcontentsline{toc}{chapter}{\numberline{}謝辞}

\newpage

\addcontentsline{toc}{chapter}{\numberline{}参考文献}
\renewcommand{\bibname}{参考文献}

%% 参考文献に jbibtex を使う場合
\bibliographystyle{junsrt}
\bibliography{ref.bib}
%% [compile] jbibtex sample; platex sample; platex sample;

%% 参考文献を直接ファイルに含めて書く場合
\begin{thebibliography}{1}
\bibitem{Bibunsho}
奥村 晴彦, 黒木 裕介.
\newblock LaTeX2ε美文書作成入門 改訂第7版.
\newblock 技術評論社, 2017.

\bibitem{ScienceResearchWriting}
Hilary Glasman-Deal.
\newblock Science Research Writing: A Guide for Non-Native Speakers of English.
\newblock Imperial College Press, 2009.
\end{thebibliography}

\end{document}
