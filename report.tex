%%
% このファイルは筑波大学情報学群情報科学類の卒業研究論文のサンプルです。
% このファイルを書き換えて、このサンプルと同様の書式の論文をLaTeXを使って
% 作成できます。
%
% OSやLaTeXの設定によっては漢字コードや改行コードを変更する必要があります。
%%
\documentclass[a4paper,11pt]{jreport}

%%【PDF, PostScript, JPEG, PNG等の画像の貼り込み】
%% dvipdfmx を使う場合
\usepackage[dvipdfmx]{graphicx}
%% dvipdfmx を使ってPDFの「しおり」を付ける場合
%%\usepackage[dvipdfmx,bookmarks=true,bookmarksnumbered=true,bookmarkstype=toc]{hyperref} \usepackage{pxjahyper}
\usepackage{ulem}
\usepackage{times} % use Times font instead of default one

\setcounter{tocdepth}{3}
\setcounter{page}{-1}

\setlength{\oddsidemargin}{0.1in}
\setlength{\evensidemargin}{0.1in}
\setlength{\topmargin}{0in}
\setlength{\textwidth}{6in}
%\setlength{\textheight}{10.1in}
\setlength{\parskip}{0em}
\setlength{\topsep}{0em}

%% タイトル生成用パッケージ(重要)
\usepackage{coins}

%% タイトル
\title{進化的計算による \ 輻輳制御アルゴリズムの探索手法の提案}
%% 著者
\author{岡部 純弥}
%% 指導教員
\advisor{岡 瑞起, 阿部 洋丈}

%% 年度と主専攻名
\fiscalyear{2023}
\majorfield{ソフトウェアサイエンス主専攻}

\begin{document}
\maketitle
\thispagestyle{empty}
\newpage

\thispagestyle{empty}
\vspace*{20pt plus 1fil}
\parindent=1zw
\noindent
%%
%% 論文の要旨
%%
\begin{center}
{\Large \bf 要  旨}
\vspace{2cm}
\end{center}

優れた輻輳制御アルゴリズムを発見することは難しい。主な理由として、コンピュータネットワークの構造が時々刻々と変化することが挙げられる。つまり、特定のネットワーク構造化で最適なアルゴリズムを探索しても、時間経過につれ、良いパフォーマンスを発揮できなくなってしまう。

そこで、大規模言語モデルと進化アルゴリズムを用いることで任意のネットワーク環境下で最適化を行う探索手法を提案する。既存の探索手法では、探索空間の制約が厳しかったものの、大規模言語モデルを用いることで、この問題を解決できた。

TODO: 事実確認

実際にネットワークシミュレータを用いた実験を行い、ベンチマークを超える輻輳制御アルゴリズムを発見できた。

%%%%%
\par
\vspace{0pt plus 1fil}
\newpage

\pagenumbering{roman} % I, II, III, IV
\tableofcontents
\listoffigures
%\listoftables

\pagebreak \setcounter{page}{1}
\pagenumbering{arabic} % 1,2,3

\chapter{序論}

\section{研究背景}

TCP/IPネットワークでは、トランスポート層で輻輳制御が行われている。輻輳制御の難しさとして、限られた観測可能な値から、観測不可能な状態を推定しなければならないことが挙げられる。
つまり、限られた観測値から、輻輳が発生しているのかを判断したうえで、パケット送信量を制御しなければなければならない。

さらに、ネットワーク構造が変化し続けているため、支配的なアルゴリズムが存在することもなく、数年に一度はパラダイムシフトが起こっている。

現在の輻輳制御アルゴリズムの探索、発見は、ヒューリスティックに行われている側面があり、人的リソースを割き続けなければらない。つまり、ネットワーク構造の変化に柔軟に対応可能な探索手法の発見が求められている。

\newpage

\section{本論文の構成}

第2章の前半では、輻輳制御アルゴリズムの概要や、その発展について述べる。
第2章の後半では、近年の大規模言語モデルの発展や、その応用について概説する。特に、輻輳制御アルゴリズムの探索手法として、大規模言語モデルを用いることの可能性について述べる。
第3章では、これらの関連研究を踏まえたうえでの仮説、および提案手法について述べる。
第4章では、仮説を検証するために提案手法の実装、実験の詳細について述べる。
第5章では、実験結果を示し、その結果に対する考察を行う。
第6章では、本論文のまとめと今後の課題について述べる。

\newpage

\chapter*{謝辞}
\addcontentsline{toc}{chapter}{\numberline{}謝辞}

本研究を行うにあたって、研究テーマの決定、研究内容の議論、外部発表の機会の提供など、様々な形でご指導を頂いた岡瑞起准教授、阿部洋丈准教授の両先生に深く感謝を申し上げます。
また、同研究プロジェクトの中で議論した、岡研究室の矢内千陽さん、OSSS研究室の佐藤創太さんの両氏にも感謝を申し上げます。
ネットワークシミュレータの取り扱いや実験環境に関して、OSSS研OBの森越さんには多大なるご協力を頂きました。
実験用のマシンのセットアップ等に関して、OSSS研の山本さん、広瀬さんには大変お世話になりました。
この場を借りて、深く御礼申し上げます。

そして、日々の研究活動やミーティングにおいて、様々な形でご協力を頂いた岡研究室の皆様、OSSS研究室の皆様にもあらためて感謝申し上げます。

\newpage

\addcontentsline{toc}{chapter}{\numberline{}参考文献}
\renewcommand{\bibname}{参考文献}

\bibliographystyle{junsrt}
\bibliography{ref.bib}

\end{document}
